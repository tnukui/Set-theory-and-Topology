\documentclass{jsarticle}


\usepackage{ascmac}
\usepackage[top=30truemm,bottom=30truemm,left=25truemm,right=25truemm]{geometry}
\usepackage{amsfonts}
\usepackage{amsmath,amssymb}
\usepackage{cases}
\begin{document}

\title{集合と位相}
\author{nukui}
\date{\today}
\maketitle

\part{集合と写像}
\section{集合とは}
\subsection{}
\begin{enumerate}
\item 成り立つ。$\because$ Yに含まれる要素は全てXに含まれる。
\item 成り立つ。$\because$3はWに含まれるがZに含まれない。
\item 成り立つ。$\because$4はVに含まれるが、Yに含まれない。
\item 成り立たない。$\because$4はVに含まれるがXには含まれない。
\item 成り立たない。$\because$ 1はXに含まれるがWに含まれない。
\item 成り立たない。$\because$ Vの全ての要素はWに含まれる。
\item 成り立つ。$\because$ Vの全ての要素はZに含まれる。
\item 成り立つ。 $\because$ 3はXに含まれるがZに含まれない。
\item 成り立たない。 $\because$ Yに含まれる全ての要素はZに含まれる。
\item 成り立たない。$\because$ 3はWに含まれるがYには含まれない。
\end{enumerate}

\subsection{}
\begin{enumerate}
\item D
\item B
\item C,E,F
\item B,D
\end{enumerate}

\subsection{}
\begin{enumerate}
\item 成り立たない。
\item 成り立つ。
\item 成り立つ。
\item 成り立つ。
\item 成り立たない。
\item 成り立つ。
\end{enumerate}

\subsection{}
集合Aが1個の元から成るとき、部分集合はAと$\emptyset$の2通り。よって$n=1$のとき、命題は成り立つ。\\
集合Aがn個の元から成り、その部分集合は全部で$2^n$個から成ると仮定する。
今、集合Aに元$X(X\notin A)$を一つ加え、$n+1$個の元から成る集合$B(B=A\cup\{X\})$を考える。
集合Bの部分集合は、
\begin{enumerate}
\item 集合Aの部分集合と一致。($2^n$個)
\item 集合Aの部分集合に元Xを加えたものに一致。($2^n$個)
\end{enumerate}
のいずれかである。よって、集合Bの部分集合の個数は$2^n+2^n=2^{n+1}$個になる。以上より、すべての自然数$n$で命題は成り立つ。

\section{集合の演算}
\subsection{}
意味を考えれば、確かに成り立つことがわかる。

\subsection{}
\begin{enumerate}
\item
\begin{align*}
(A-B)\cup(A\cap B)&=\{x|x\in(A-B) またはx\in(A\cap B)\}\\
&=\{x|(x\in A かつ x\notin B) または (x\in A かつ x\in B)\}\\
&=\{x|x\in A かつ (x\notin B または x\in B)\}\\
&=\{x|x\in A\}\\
&=A
\end{align*}


\item
\begin{align*}
(A-B)\cup B &=\{x| x\in (A-B) または x \in B\}\\
&=\{ x | (x \in A かつ x\notin B) または x\in B \}\\
&=\{x | (x\in Aまたは x\in B) かつ( x\notin B または x\in B)\}\\
&=\{x| (x\in Aまたは x\in B)\}\\
&=A\cup B
\end{align*}


\item
\begin{align*}
B\cap(A-B)&=\{x|x\in B かつ x \in(A-B)\}\\
&=\{x|x\in B かつ( x \in A かつ x\notin B)\}\\
&=\emptyset
\end{align*}
\end{enumerate}

\subsection{}
\begin{enumerate}
\item
$A_1 \subset A$を仮定する。
\[x \in A_1 かつ x\notin B \Longrightarrow x\in A かつ x\notin B\]
なので、$x\in A_1-B$とすると、$x\in A-B$が示せる。つまり、$A_1-B \subset A-B$。
\item
$B_1 \subset B$を仮定する。
\[x \in A かつ x\notin B \Longrightarrow x\in A かつ x\notin B_1\]
なので、$x\in A-B$とすると、$x\in A-B_1$が示せる。つまり、$A-B \subset A-B_1$。
\end{enumerate}

\subsection{}
\begin{align*}
A-B&=\{x|x\in A かつ x\notin B\}\\
&=\{x|x\in A かつ (x\notin A または x\notin B)\}\\
&=\{x|x\in A かつ x\notin A\cap B\}\\\\
A&=\{x|x\in A\}
\end{align*}
なので、
\[A-B=A \Longleftrightarrow  A-B \supset A \Longleftrightarrow A\cap B = \emptyset\]
となり、$A-B=A$と、$A\cap B=\emptyset$が同値であることを示せた。

\subsection{}
\begin{enumerate}
\item
定義から、
\begin{align*}
A\cup B &= \{x | x\in A または x\in B\}\\
B&= \{x| x\in B\}
\end{align*}である。
ここで、$A\subset B$を仮定すると、$A\cup B = \{x | x\in B\}=B$となる。逆に、$A\cup B= B$を仮定すると、$A\cup B \subset B$より、$\forall x [ x\in A \Longrightarrow x\in B]$となるので、$A\subset B$。\\

\item
定義から、
\begin{align*}
A\cap B &= \{x | x\in A かつ x\in B\}\\
A&= \{x| x\in A\}
\end{align*}である。
ここで、$A\subset B$を仮定すると、$A\cap B = \{x | x\in A\}=A$となる。逆に、$A\cap B= A$を仮定すると、$ A\cap B \supset A$より、$\forall x [ x\in A \Longrightarrow x\in B]$となるので、$A\subset B$。\\

\item 定義から
\[A-B=\{x|x\in A かつ x\notin B\}\]
である。ここで、$A\subset B$を仮定すると、$\forall x [ x\in A \Longrightarrow x\in B]$なので、$A-B=\emptyset$が成り立つ。逆に、$A-B=\emptyset$を仮定すると、$\forall x [ x\in A \Longrightarrow x\in B]$になるので、$A\subset B$が成り立つ。\\

\item 定義から
\begin{align*}
A\cup (B-A)&=\{x|x\in A または x\in (B-A)\}\\
&=\{x|x\in A または (x\in B かつ x\notin A)\}\\
&=\{x|x\in A または x\in B\}\\
&= A\cup B
\end{align*}
よって、1と本質的に同じ問題なので、成立する。\\
\item 定義から
\begin{align*}
B-(B-A)&=\{x| x\in B かつ x\notin (B-A)\}\\
&=\{x| x\in B かつ (x\notin B または x\in A)\}\\
&=\{x| x \in B かつ x\in A\}\\
&= A\cap B
\end{align*}
よって、本質的に2と同じ問題なので、成立する。
\end{enumerate}

\subsection{}
\begin{enumerate}
\item
\begin{align*}
(A\cup B) \cap (A\cup C) \cap(B\cup C) 
&=((A\cup B) \cap (A\cup C)) \cap(B\cup C) \\
&=((A\cup B) \cap A) \cup ((A\cup B) \cap C)) \cap(B\cup C) \\
&=(A\cup ((A\cap C) \cup (B\cap C)))\cap(B\cup C) \\
&=(A \cup (B\cap C))\cap(B\cup C) \\
&=(A\cap(B\cup C)) \cup ((B\cap C)\cap (B\cup C))\\
&=((A\cap B)\cup (A\cap C)) \cup (B\cap C)\\
&=(A\cap B)\cup (A\cap C) \cup (B\cap C)
\end{align*}
\item 1の結果を用いる。
\begin{align*}
&\quad(A\cup B) \cap (A\cup C) \cap(A\cup D)\cap (B\cup C) \cap (B\cup D) \cap(C\cup D)\\
&=((A\cup B) \cap (A\cup C)\cap(B\cup C) )\cap((A\cup D)\cap (B\cup D)\cap(C\cup D))\\
&=((A\cap B)\cup (A\cap C) \cup (B\cap C))\cap (D\cup (A\cap B \cap C))\\
&=(((A\cap B)\cup (A\cap C) \cup (B\cap C))\cap D)\cup(((A\cap B)\cup (A\cap C) \cup (B\cap C))\cap(A\cap B \cap C))\\
&=((A\cap B\cap D)\cup (A\cap C\cap D) \cup (B\cap C\cap D))\cup (A\cap B \cap C)\\
&=(A\cap B \cap C)\cup(A\cap B\cap D)\cup(A\cap C\cap D)\cup (B\cap C\cap D)
\end{align*}
\end{enumerate}


\section{ド・モルガンの法則}
\subsection{}
\begin{enumerate}
\item
\begin{align*}
(A^c)^c&=(X-A)^c\\
&=\{x | x\in X - (X-A)\}\\
&=\{x\in X|x \notin X-A\}\\
&=\{x\in X|「x \in X かつ x \notin A」ではない\}\\
&=\{x\in X| x\notin X または x \in A\}\\
&=\{x\in X | x \in A\}\\
&= A
\end{align*}


\item
\[X^c = X-X =\{x| x\in X かつ x\notin X\}= \emptyset \]

\item
\[\emptyset^c = X-\emptyset = \{x | x\in X かつ x\notin \emptyset\} = X\]

\item
\begin{align*}
A\cup A^c &= \{x| x\in A または x\in A^c\}\\
&=\{x \in X | x \in A または x\notin A\}\\
&=X
\end{align*}

\item
\[A\cap A^c = (A^c)^c \cap A^c = (A^c \cup A)^c = X^c = \emptyset\]

\item
\begin{align*}
A-B&=\{x\in X | x\in A かつ x\notin B\}\\
&=\{x | x\in A かつ x\in X-B\}\\
&=\{x| x\in A かつ x\in B^c\}\\
&= A\cap B^c
\end{align*}

\item

\[(A^c\cap B^c)^c = (A^c)^c \cup (B^c)^c = A\cup B\]

\end{enumerate}

\subsection{}
\begin{enumerate}
\item
\begin{align*}
(A\cup B)-C &= \{x | x \in A\cup B かつ x\notin C\}\\
&=\{x| (x \in A または x\in B) かつ x\notin C\}\\
&=\{x| (x\in A かつ x\notin C) または (x\in B かつ x\notin C)\}\\
&=\{x| x\in A-C または x\in B-C\}\\
&=(A-C)\cup (B-C)
\end{align*}

\item
\begin{align*}
(A\cap B)-C &= \{x | x \in A\cap B かつ x\notin C\}\\
&=\{x| (x \in A かつ x\in B) かつ x\notin C\}\\
&=\{x| (x\in A かつ x\notin C) かつ (x\in B かつ x\notin C)\}\\
&=\{x| x\in A-C かつ x\in B-C\}\\
&=(A-C)\cap (B-C)
\end{align*}

\end{enumerate}

\subsection{}
\begin{enumerate}
\item
\begin{enumerate}
\item
\begin{align*}
A\circ B &= (A-B)\cup(B-A)\\
&=(B-A)\cup (A-B)\\
&= B\circ A
\end{align*}


\item
\begin{align*}
(A\circ B)\circ C &= ((A\circ B)-C) \cup (C-(A\circ B))\\
&=(((A-B)\cup(B-A))-C)\cup(C-((A-B)\cup(B-A)))\\
&=(((A-B)-C)\cup((B-A)-C))\cup((C-(A-B)) \cap (C - (B-A)))\\
&=((A-(B\cup C)) \cup (B-(A\cup C))) \cup (((C-A)\cup (C\cap B))\cap((C-B)\cup(C\cap A)))\\
&=((A-(B\cup C)) \cup (B-(A\cup C))) \cup \\
&\qquad((C-A)\cap(C-B)) \cup( (C-A)\cap(C\cap A))\cup((C\cap B)\cap(C-B))\cup((C\cap B)\cap(C\cap A))\\
&=((A-(B\cup C)) \cup (B-(A\cup C)))\cup((C-(A\cup B)) \cup \emptyset \cup \emptyset \cup (A\cap B \cap C))\\
&=(A-(B\cup C)) \cup (B-(C\cup A))\cup(C-(A\cup B)) \cup  (A\cap B \cap C)
\end{align*}

ここで、
\begin{align*}
C-(A-B)&=\{x|x\in C かつ x\notin (A-B)\}\\
&=\{x|x\in C かつ (x\notin A または x\in B))\}\\
&=\{x|(x\in C かつ x\notin A) または (x\in C かつx\in B)\}\\
&=(C-A)\cup(C\cap B)
\end{align*}
という結果を、3行目から4行目への変形で用いた。また、
\begin{align*}
(A\cup B)\cap(C\cup D)&=(A\cap (C\cup D))\cup(B\cap (C\cup D))\\
&=(A\cap C)\cup (A\cap D) \cup (B\cap C)\cup(B\cap D)
\end{align*}
という結果を、4行目から(5,6行目)への変形で用いた。\\
さて、$(A\circ B)\circ C=(A-(B\cup C)) \cup (B-(C\cup A))\cup(C-(A\cup B)) \cup  (A\cap B \cap C)$と変形できるので、$(A\circ B)\circ C$は、A,B,Cが全く同等で、任意の二つを入れ替えても同じ値であることがわかる。よって、$(A\circ B)\circ C=A\circ (B\circ C)$。
\item
\[A\circ A= (A-A)\cup (A-A)=\emptyset\]

\item
\[A\circ \emptyset = (A-\emptyset)\cup(\emptyset-A)=A\]

\end{enumerate}
\item
\begin{align*}
&\qquad A\circ X=B\\
&\iff (A\circ X)-B=\emptyset かつB-(A\circ X)=\emptyset\\
&\iff ((A\circ X)-B) \cup (B-(A\circ X))=\emptyset\\
&\iff ((A\circ X)\circ B)=\emptyset\\
&\iff ((A\circ B)\circ X)=\emptyset\\
&\iff ((A\circ B)-X)\cup (X-(A\circ B))=\emptyset\\
&\iff ((A\circ B)-X)=\emptyset かつ (X-(A\circ B))=\emptyset\\
&\iff A\circ B=X
\end{align*}
ここで、$((A\circ X)\circ B)=\emptyset\iff((A\circ B)\circ X)=\emptyset$という変形には、1の結果を用いた。
以上より、集合A,Bを任意に与えたとき、$A\circ X=B$を満足する集合$X$が$A\circ B$と表せるので、$X$はただ一つ存在することが示せた。
\end{enumerate}

\section{直積集合}
\subsection{}
\begin{enumerate}
\item
\begin{align*}
A\times(B\cup C)&= \{(x,y)|x\in A かつy\in(B\cup C)\}\\
&=\{(x,y)|(x\in A かつ y\in B)または(x\in A かつ y\in C)\}\\
&=\{(x,y)|(x,y)\in (A\times B) または (x,y)\in(A\times C)\}\\
&=(A\times B) \cup (A\times C)
\end{align*}

\item
\begin{align*}
A\times(B\cap C)&= \{(x,y)|x\in A かつy\in(B\cap C)\}\\
&=\{(x,y)|(x\in A かつ y\in B)かつ(x\in A かつ y\in C)\}\\
&=\{(x,y)|(x,y)\in (A\times B) かつ (x,y)\in(A\times C)\}\\
&=(A\times B) \cap (A\times C)
\end{align*}

\item
\begin{align*}
(A\cup B)\times C&=\{(x,y)|x\in (A\cup B) かつ y\in C\}\\
&=\{(x,y)|(x\in A かつ y\in C)または(x\in B かつ y\in C)\}\\
&=\{(x,y)|(x,y)\in A\times C または (x,y)\in B \times C\}\\
&=(A\times C)\cup(B\times C)
\end{align*}

\item
\begin{align*}
(A\cap B)\times C&=\{(x,y)|x\in (A\cap B) かつ y\in C\}\\
&=\{(x,y)|(x\in A かつ y\in C)かつ(x\in B かつ y\in C)\}\\
&=\{(x,y)|(x,y)\in A\times C かつ (x,y)\in B \times C\}\\
&=(A\times C)\cap(B\times C)
\end{align*}

\end{enumerate}

\subsection{}
\begin{align*}
(X\times Y)-(A\times B)&=\{(s,t)|(s,t)\in(X\times Y)かつ(s,t)\notin(A\times B)\}\\
&=\{(s,t)|(s\in(X-A) かつ t\in Y)\\
&\qquad\qquad または(s\in X かつ t\in(Y-B))\}\\
&=\{(s,t)|(s,t)\in ((X-A)\times Y) または (s,t)\in(X\times (Y-B))\}\\
&=((X-A)\times Y)\cup(X\times(Y-B))
\end{align*}

\section{写像}
\subsection{}
\begin{enumerate}
\item
\[(f\circ g)(x)=f(x^2+1)=x^2+3\]
\item
\[(g\circ f)(x)=g(x+2)=(x+2)^2+1=x^2+4x+5\]
\item
\[(f\circ f)(x)=f(x+2)=x+4\]
\item
\[(g\circ g)(x)=g(x^2+1)=(x^2+1)^2+1=x^4+2x^2+2\]
\end{enumerate}

\subsection{}
\begin{enumerate}
\item
\begin{align*}
(\bigcup_{\lambda\in\Lambda}A_\lambda)\cap B&=\{x| x\in\bigcup_{\lambda\in\Lambda}A_\lambda かつ x\in B\}\\
&=\{x|\exists \lambda\in\Lambda[x\in A_{\lambda}]かつx\in B\}\\
&=\{x|\exists \lambda\in\Lambda[x\in A_{\lambda} かつ x\in B]\}\\
&=\{x|\exists \lambda\in\Lambda[x\in (A_{\lambda} \cap B)]\}\\
&=\bigcup_{\lambda\in\Lambda}(A_{\lambda}\cap B)
\end{align*}

\item
\begin{align*}
(\bigcap_{\lambda\in\Lambda}A_\lambda)\cup B&=\{x| x\in\bigcap_{\lambda\in\Lambda}A_\lambda または x\in B\}\\
&=\{x|\forall \lambda\in\Lambda[x\in A_{\lambda}]またはx\in B\}\\
&=\{x|\forall \lambda\in\Lambda[x\in A_{\lambda} または x\in B]\}\\
&=\{x|\forall \lambda\in\Lambda[x\in (A_{\lambda} \cup B)]\}\\
&=\bigcap_{\lambda\in\Lambda}(A_{\lambda}\cup B)
\end{align*}
\end{enumerate}

\subsection{}
\begin{enumerate}
\item
\begin{align*}
(\bigcup_{\lambda\in\Lambda}A_{\lambda})^c&=(X-\bigcup_{\lambda\in\Lambda}A_{\lambda})\\
&=\{x|x\in X かつ 「\exists\lambda\in\Lambda [x\in A_\lambda]ではない」\}\\
&=\{x|x\in X かつ \forall \lambda\in\Lambda[x\notin A_\lambda]\}\\
&=\{x|\forall \lambda\in\Lambda[x\in X かつ x\notin A_\lambda]\}\\
&=\bigcap_{\lambda\in\Lambda}(A_\lambda^c)
\end{align*}

\item
\begin{align*}
(\bigcap_{\lambda\in\Lambda}A_{\lambda})^c&=(X-\bigcap_{\lambda\in\Lambda}A_{\lambda})\\
&=\{x|x\in X かつ 「\forall\lambda\in\Lambda [x\in A_\lambda]ではない」\}\\
&=\{x|x\in X かつ \exists \lambda\in\Lambda[x\notin A_\lambda]\}\\
&=\{x|\exists \lambda\in\Lambda[x\in X かつ x\notin A_\lambda]\}\\
&=\bigcup_{\lambda\in\Lambda}(A_\lambda^c)
\end{align*}
\end{enumerate}

\subsection{}
\begin{enumerate}
\item
\begin{align*}
f(\bigcup_{\lambda\in\Lambda}A_\lambda)&=\{y\in Y| \exists x \in(\bigcup_{\lambda\in\Lambda}A_\lambda)[f(x)=y]\} \\
&=\{y\in Y|\exists \lambda \in \Lambda[\exists x \in A_\lambda[f(x)=y]]\}\\
&=\{y\in Y|\exists \lambda \in \Lambda[y\in f(A_\lambda)]\}\\
&=\bigcup_{\lambda \in \Lambda}f(A_\lambda)
\end{align*}
\item
\[y\in f(\bigcap_{\lambda\in\Lambda}A_\lambda)\]と仮定すると、\[\exists x \in(\bigcap_{\lambda\in\Lambda}A_\lambda)[f(x)=y]\]が言える。これは、
\[\exists x \in X[\forall \lambda \in \Lambda[x \in A_\lambda かつ f(x)=y]]\]
と同値。このとき、
\[\forall \lambda \in \Lambda[\exists x \in A_\lambda [f(x)=y]]\]
がいえる\footnote{逆は必ずしも真ではない。つまり$\forall \lambda \in \Lambda[\exists x \in A_\lambda [f(x)=y]]$だからといって、$\exists x \in X[\forall \lambda \in \Lambda[x \in A_\lambda かつ f(x)=y]]$はいえない。}ので、
\[y\in \bigcap_{\lambda\in\Lambda}f(A_{\lambda})\]

\item
\begin{align*}
f^{-1}(\bigcup_{\mu\in M}B_{\mu})&=\{x\in X| f(x) \in(\bigcup_{\mu\in M}B_{\mu})\} \\
&=\{x\in X| \exists \mu \in M [f(x) \in B_{\mu}]\}\\
&=\{x\in X| \exists \mu \in M [x\in f^{-1}(B_{\mu})]\}\\
&=\bigcup_{\mu\in M}f^{-1}(B_{\mu})
\end{align*}

\item
\begin{align*}
f^{-1}(\bigcap_{\mu\in M}B_{\mu})&=\{x\in X|f(x)\in(\bigcap_{\mu\in M}B_{\mu}) \}\\
&=\{x\in X | \forall \mu\in M[f(x)\in B_{\mu}]\}\\
&=\{x\in X | \forall \mu\in M [x\in f^{-1}(B_{\mu})]\}\\
&=\bigcap_{\mu\in M}f^{-1}(B_{\mu})
\end{align*}
\end{enumerate}

\subsection{}
$n=2$のとき
\[A_1 \cup A_2 =A_1\cup A_2\]
となり、主張は正しい。
$n=m$のとき、主張が正しいと仮定する。つまり、
\[\bigcap_{1\leqq i<j\leqq m}(A_i \cup A_j )=\bigcup_{1\leqq i\leqq m}(A_1\cap \cdots\cap A_{i-1}
\cap A_{i+1} \cap \cdots \cap A_{m})\]
が成り立つと仮定すると、
\begin{align*}
&\quad\bigcap_{1\leqq i<j\leqq m+1}(A_i \cup A_j )\\
&=(\bigcap_{1\leqq i<j\leqq m}(A_i \cup A_j ))\cap (\bigcap_{1\leqq i\leqq m}(A_{i}\cup A_{m+1}))\\
&=(\bigcup_{1\leqq i\leqq m}(A_1\cap \cdots\cap A_{i-1}\cap A_{i+1} \cap \cdots \cap A_{m}))\cap((\bigcap_{1\leqq i\leqq m}A_{i})\cup A_{m+1})\\
&=\bigcup_{1\leqq i\leqq m}(A_1\cap \cdots\cap A_{i-1}\cap A_{i+1} \cap \cdots \cap A_{m}\cap((\bigcap_{1\leqq i\leqq m}A_{i})\cup A_{m+1}))\\
&=(\bigcap_{1\leqq i\leqq m}A_{i})\cup(\bigcup_{1\leqq i\leqq m}(A_1\cap \cdots\cap A_{i-1}\cap A_{i+1} \cap \cdots \cap A_{m+1}))\\
&=\bigcup_{1\leqq i\leqq m+1}(A_1\cap \cdots\cap A_{i-1}
\cap A_{i+1} \cap \cdots \cap A_{m+1})
\end{align*}
となり、$n=m+1$のときも成り立つ。以上より、$n\geqq 2$のすべての自然数$n$で主張は成り立つ。


\subsection{}
\begin{enumerate}
\item
\[x\in\liminf_{n\to\infty}E_{n}\]
と仮定すると、定義より
\[\exists k \geqq 1 [\forall n\geqq k[x\in E_{n}]]\]
が成り立つ。よって、
\[\forall s \geqq 1[\exists k\geq 1[(m \geq k かつ m\geq s)\Longrightarrow x\in E_{m}]]\]
となるので、
\[\forall s \geq 1[\exists m\geq s[x\in E_{m}]]\]
定義より、\[x\in\limsup_{n\to\infty}E_{n}\]

\item
\[x\in\liminf_{n\to\infty}A_{n}\]
と仮定すると、定義より、
\[\exists k\geqq 1[\forall n\geqq k[x\in A_{n}]]\]
ここで、$\forall n\in N[A_n \subset B_n]$なので、
\[\exists k\geqq 1[\forall n\geqq k[x\in B_{n}]]\]
が言える。結局、
\[x\in\liminf_{n\to\infty}B_{n}\]
$\limsup$の場合も同様に示せる。
\item
\begin{align*}
\limsup_{n\to\infty}(A_{n}\cup B_{n})&=\{x|\forall k\geqq 1[\exists n\geqq k[x\in(A_{n}\cup B_{n})]]\}\\
&=\{x|\forall k\geqq 1[\exists n\geqq k[x\in A_{n} または x\in B_{n}]]\}\\
&=\{x|\forall k\geqq 1[(\exists n\geqq k[x\in A_{n}]) または (\exists m \geqq k[x\in B_{m}])]\}\\
&=\{x|(\forall k\geqq 1[\exists n\geqq k[x\in A_{n}]]) または (\forall s\geqq 1[\exists m \geqq s[x\in B_{m}]])\}\\
&=\{x|x\in(\limsup_{n\to\infty}A_{n})\cup (\limsup_{n\to\infty}B_{n})\}\\
&=(\limsup_{n\to\infty}A_{n})\cup (\limsup_{n\to\infty}B_{n})
\end{align*}
ここで、$3行目\Longrightarrow4行目$を示すには、「4行目でない$\Longrightarrow$3行目でない」を示せば良い。
\item
\begin{align*}
\liminf_{n\to\infty}(A_{n}\cap B_{n})&=\{x|\exists k\geqq 1[\forall n\geqq k[x\in(A_{n}\cap B_{n})]]\}\\
&=\{x|\exists k\geqq 1[\forall n\geqq k[x\in A_{n} かつ x\in B_{n}]]\}\\
&=\{x|\exists k\geqq 1[(\forall n\geqq k[x\in A_{n}]) かつ (\forall m \geqq k[x\in B_{m}])]\}\\
&=\{x|(\exists k\geqq 1[\forall n\geqq k[x\in A_{n}]]) かつ (\exists s\geqq 1[\forall m \geqq s[x\in B_{m}]])\}\\
&=\{x|x\in(\liminf_{n\to\infty}A_{n})\cap (\liminf_{n\to\infty}B_{n})\}\\
&=(\liminf_{n\to\infty}A_{n})\cap (\liminf_{n\to\infty}B_{n})
\end{align*}
\end{enumerate}
\subsection{}
\begin{enumerate}
\item
 \\各$n\in N$に対して、$E_{n}\subset E_{n+1}$のとき、$\bigcap_{n=k}^{\infty}E_{n}=E_{k}$なので、
\begin{align*}
\lim_{n\to\infty}E_{n}&=\bigcup_{k=1}^{\infty}\bigcap_{n=k}^{\infty}E_{n}\\
&=\bigcup_{k=1}^{\infty}E_{k}
\end{align*}
\item
 \\各$n\in N$に対して、$E_{n}\supset E_{n+1}$のとき、$\bigcup_{n=k}^{\infty}E_{n}=E_{k}$なので、
\begin{align*}
\lim_{n\to\infty}E_{n}&=\bigcap_{k=1}^{\infty}\bigcup_{n=k}^{\infty}E_{n}\\
&=\bigcap_{k=1}^{\infty}E_{k}
\end{align*}

\end{enumerate}
\subsection{}
\begin{enumerate}
\item
問5.6(3)より、
\[\lim_{n\to\infty}(A_n \cup B_n)=\limsup_{n\to\infty}(A_n \cup B_n)=\limsup_{n\to\infty}A_n \cup \limsup_{n\to\infty}B_n=\lim_{n\to\infty}A_n\cup\lim_{n\to\infty}B_n\]
\item
問5.6(4)より、
\[\lim_{n\to\infty}(A_n \cap B_n)=\liminf_{n\to\infty}(A_n \cap B_n)=\liminf_{n\to\infty}A_n \cap \liminf_{n\to\infty}B_n=\lim_{n\to\infty}A_n\cap\lim_{n\to\infty}B_n\]
\end{enumerate}

\subsection{}
\begin{enumerate}
\item
\begin{align*}
\limsup_{n\to\infty}E_{n}&=\{x|\forall k\geqq 1[\exists n \geqq k[x\in E_{n}]]\}\\
&=\{x|\forall k\geqq 1[\exists n[2n-1 \geqq k かつ (x\in E_{2n} または x\in E_{2n-1})]]\}\\
&=\{x|\forall k\geqq 1[\exists n [2n-1 \geqq k かつ (x \in A または x\in B)]]\}\\
&=\{x|x\in A または x\in B\}\\
&=A\cup B
\end{align*}
\item
\begin{align*}
\liminf_{n\to\infty}E_{n}&=\{x|\exists k\geqq 1[\forall n \geqq k[x\in E_{n}]]\}\\
&=\{x|\exists k\geqq 1[\forall n [2n-1\geqq k \Longrightarrow (x\in E_{2n-1}かつx\in E_{2n})]]\}\\
&=\{x|\exists k\geqq 1[\forall n [2n-1\geqq k \Longrightarrow (x\in A かつx\in B)]]\}\\
&=\{x|x\in A かつx\in B\}\\
&=A\cap B
\end{align*}
\end{enumerate}



\part{濃度の大小と二項関係}
\section{全射・単射}
\subsection{}
\begin{enumerate}
\item
$f:A\to B$が単射と仮定する。
\begin{enumerate}
\item
\[y\in f(A_1)\cap f(A_2)\]
と仮定すると、$y=f(x)$となる$x$が存在し、$x\in A_1$かつ$x\in A_2$となる。(もしも$x\notin A_1$または$x\notin A_2$とすると、$f$が単射でないことになってしまう。)よって、
\[y\in f(A_1 \cap A_2)\]\\
\item
任意の$x$について$f^{-1}(f(x))=x$なので、$x\in f^{-1}(f(A_1))$と仮定すると$x\in A_1$である。\\
\item
\[y\in f(A_1 - A_2)\]
と仮定すると、$f(x)=y$となる$x\in A_1 - A_2$がただ一つ存在する。よって、$y\in f(A_1)$かつ$y\notin f(A_2)$である。
\end{enumerate}
\item
\[f:A\to B\]
が全射とする。$y\in B_1$と仮定すると、$f(x)=y$となる$x\in f^{-1}(B_1)$が存在する。よって、$y\in f(f^{-1}(B_1))$。
\end{enumerate}
\subsection{}
\begin{enumerate}
\item
$g\circ f$が単射と仮定すると、
\[\forall x_1, x_2\in A[x_1\neq x_2 \Longrightarrow g\circ f(x_1)\neq g\circ f(x_2) ]\]
ここで、$f(x_1)=f(x_2)\Longrightarrow g\circ f(x_1)=g\circ f(x_2)$なので、
\[\forall x_1, x_2\in A[x_1\neq x_2 \Longrightarrow f(x_1)\neq f(x_2) ]\]
が成り立つ。よって、$f$は単射である。
\item
$g\circ f$が全射と仮定すると、
\[\forall c\in C[\exists a\in A[g\circ f(a)=c]]\]
ここで、$\forall a\in A[f(a)\in B]$なので、
\[\forall c\in C[\exists b\in B[g(b)=c]]\]
\end{enumerate}

\subsection{}
\[\forall n\in \mathbb{N}[\forall x\in X[h^n(x)\neq x]]\]
と仮定すると、任意の$n\in\mathbb{N}, x\in X$について、$x,h(x),h^2(x),\dots,h^n(x)$は互いに異なる値となる。つまり、$h:X\to X$より、Xに無限個の元が存在することになってしまい、矛盾。
よって、
\[\exists n\in \mathbb{N}[\exists x\in X[h^n(x)= x]]\]

\subsection{}
たとえば、$y=\frac{d-c}{b-a}x+\frac{bc-ad}{b-a}$は、条件を満たす。

\subsection{}
$f$は全射かつ単射である。実際、
\[
\begin{cases}
y=\frac{1}{2}ならば、x=0\\
y=\frac{1}{2^n}\quad(n=2,3,\dots)ならば、x=4y\\
それ以外ならば、x=y
\end{cases}
\]
というように、任意の$y$に関して、$f(x)=y$となる$x$がただ一つ存在するので、$f$は全単射。

\subsection{}
以下のように定義された写像$f$は条件を満たす。
\[f(x)=
\begin{cases}
1,  \qquad x=0\\
\frac{x}{2}, \qquad x=\frac{1}{2^n}(n=0,1,2,\dots)\\
x, \qquad x\neq 0, \frac{1}{2^n}(n=0,1,2,\dots)
\end{cases}
\]

\section{濃度の大小}
\subsection{}
\begin{enumerate}
\item
恒等写像$1_{A}:A\to A$は全単射である。
\item
$A\sim B$と仮定すると、全単射の写像$f:A\to B$が存在する。ここで、$f$の逆写像$f^{-1}$も全単射である。よって、$B\sim A$。
\item
$A\sim B$かつ$B\sim C$と仮定すると、全単射の写像$f:A\to B$と、$g:B\to C$が存在する。ここで、$g\circ f:A\to C$も全単射である。よって、$A\sim C$。
\end{enumerate}

\subsection{}
$F(A\times B, C)$の元である写像$f:A\times B\to C$が与えられた時、$F(A,F(B,C))$の元である写像$g:A\to F(B,C)$を以下のように定義することで対応させるとする。
\[(g(a))(b)=f(a,b)\]
このとき、任意の$g:A\to F(B,C)$に対して、対応する$f:A\times B\to C$がただ一つ存在する。実際、任意の$g$に対して、$f(a,b)=(g(a))(b)$という$f:A\times B\to C$がただ一つ存在する。よって、$F(A\times B, C)$の元と$F(A,F(B,C))$の元が一対一に対応付けできた。以上より、$F(A\times B, C)\sim F(A,F(B,C))$が示された。

\subsection{}
\begin{enumerate}
\item
$A\sim A'$かつ$B\sim B'$と仮定すると、全単射の写像$f_1:A\to A'$と$f_2:B\to B'$が存在する。今$g:A\times B \to A'\times B'$を$g(a,b)=(f_1(a),f_2(b))$として定義すれば、これは全単射の写像である。\\
また、$h_1\in F(A,B)$に対して、$h_2(a')=f_2(h_1(f_1^{-1}(a')))$として定義された$h_2 \in F(A',B')$を対応づけるとする。このとき、任意の$h_2 \in F(A',B')$に対して、$h_1\in F(A,B)$がただ一つ存在する。実際、任意の$h_2$に対して、$h_1(a)=f_2^{-1}(h_2(f_1(a)))$という$h_1\in F(A',B')$がただ一つ存在している。以上より、$F(A,B)\sim F(A',B')$が示された。
\item
$A\sim B$と仮定すると、全単射の写像$f:A\to B$が存在する。今、$g:\mathfrak{P}(A)\to\mathfrak{P}(B)$を、\[g(A')=\{b\in B|a\in A' かつ b=f(a)\}\]と定義する(ここで$A'\subset A$)。このとき、$g$は全単射である。実際、逆写像$g^{-1}$が以下のように定義できる。
\[g^{-1}(B')=\{a\in A|b\in B' かつ b=f(a)\}\]
\end{enumerate}


\end{document}

