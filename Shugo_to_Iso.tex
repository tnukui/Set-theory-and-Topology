\documentclass{jsarticle}


\usepackage{ascmac}
\usepackage[top=30truemm,bottom=30truemm,left=25truemm,right=25truemm]{geometry}
\usepackage{amsfonts}
\usepackage{amsmath,amssymb}
\begin{document}

\title{集合と位相}
\author{nukui}
\date{\today}
\maketitle

\part{集合と写像}
\section{集合とは}
\subsection{}
\begin{enumerate}
\item 成り立つ。$\because$ Yに含まれる要素は全てXに含まれる。
\item 成り立つ。$\because$3はWに含まれるがZに含まれない。
\item 成り立つ。$\because$4はVに含まれるが、Yに含まれない。
\item 成り立たない。$\because$4はVに含まれるがXには含まれない。
\item 成り立たない。$\because$ 1はXに含まれるがWに含まれない。
\item 成り立たない。$\because$ Vの全ての要素はWに含まれる。
\item 成り立つ。$\because$ Vの全ての要素はZに含まれる。
\item 成り立つ。 $\because$ 3はXに含まれるがZに含まれない。
\item 成り立たない。 $\because$ Yに含まれる全ての要素はZに含まれる。
\item 成り立たない。$\because$ 3はWに含まれるがYには含まれない。
\end{enumerate}

\subsection{}
\begin{enumerate}
\item D
\item B
\item C,E,F
\item B,D
\end{enumerate}

\subsection{}
\begin{enumerate}
\item 成り立たない。
\item 成り立つ。
\item 成り立つ。
\item 成り立つ。
\item 成り立たない。
\item 成り立つ。
\end{enumerate}

\subsection{}
集合Aが1個の元から成るとき、部分集合はAと$\emptyset$の2通り。よって$n=1$のとき、命題は成り立つ。\\
集合Aがn個の元から成り、その部分集合は全部で$2^n$個から成るとする。
今、集合Aに元Xを一つ加え、n+1個の元から成る集合Bを考える。
集合Bの部分集合は、
\begin{enumerate}
\item 集合Aの部分集合と一致。($2^n$個)
\item 集合Aの部分集合に元Xを加えたものに一致。($2^n$個)
\end{enumerate}
のいずれかである。よって、集合Bの部分集合の個数は$2^n+2^n=2^{n+1}$個になる。以上より、すべての自然数$n$で命題は成り立つ。

\section{集合の演算}
\subsection{}
意味を考えれば、確かに成り立つことがわかる。

\subsection{}
\begin{enumerate}
\item
\begin{align*}
(A-B)\cup(A\cap B)&=\{x|x\in(A-B) またはx\in(A\cap B)\}\\
&=\{x|(x\in A かつ x\notin B) または (x\in A かつ x\in B)\}\\
&=\{x|x\in A かつ (x\notin B または x\in B)\}\\
&=\{x|x\in A\}\\
&=A
\end{align*}


\item
\begin{align*}
(A-B)\cup B &=\{x| x\in (A-B) または x \in B\}\\
&=\{ x | (x \in A かつ x\notin B) または x\in B \}\\
&=\{x | (x\in Aまたは x\in B) かつ( x\notin B または x\in B)\}\\
&=\{x| (x\in Aまたは x\in B)\}\\
&=A\cup B
\end{align*}


\item
\begin{align*}
B\cap(A-B)&=\{x|x\in B かつ x \in(A-B)\}\\
&=\{x|x\in B かつ( x \in A かつ x\notin B)\}\\
&=\emptyset
\end{align*}
\end{enumerate}

\subsection{}
\begin{enumerate}
\item
$A_1 \subset A$を仮定する。
\[x \in A_1 かつ x\notin B \Longrightarrow x\in A かつ x\notin B\]
なので、$x\in A_1-B$とすると、$x\in A-B$が示せる。つまり、$A_1-B \subset A-B$。
\item
$B_1 \subset B$を仮定する。
\[x \in A かつ x\notin B \Longrightarrow x\in A かつ x\notin B_1\]
なので、$x\in A-B$とすると、$x\in A-B_1$が示せる。つまり、$A-B \subset A-B_1$。
\end{enumerate}

\subsection{}
\begin{align*}
A-B&=\{x|x\in A かつ x\notin B\}\\
&=\{x|x\in A かつ (x\notin A または x\notin B)\}\\
&=\{x|x\in A かつ x\notin A\cap B\}\\\\
A&=\{x|x\in A\}
\end{align*}
なので、
\[A-B=A \Longleftrightarrow  A-B \supset A \Longleftrightarrow A\cap B = \emptyset\]
となり、$A-B=A$と、$A\cap B=\emptyset$が同値であることを示せた。

\subsection{}
\begin{enumerate}
\item
定義から、
\begin{align*}
A\cup B &= \{x | x\in A または x\in B\}\\
B&= \{x| x\in B\}
\end{align*}である。
ここで、$A\subset B$を仮定すると、$A\cup B = \{x | x\in B\}=B$となる。逆に、$A\cup B= B$を仮定すると、$A\cup B \subset B$より、$\forall x [ x\in A \Longrightarrow x\in B]$となるので、$A\subset B$。\\

\item
定義から、
\begin{align*}
A\cap B &= \{x | x\in A かつ x\in B\}\\
A&= \{x| x\in A\}
\end{align*}である。
ここで、$A\subset B$を仮定すると、$A\cap B = \{x | x\in A\}=A$となる。逆に、$A\cap B= A$を仮定すると、$ A\cap B \supset A$より、$\forall x [ x\in A \Longrightarrow x\in B]$となるので、$A\subset B$。\\

\item 定義から
\[A-B=\{x|x\in A かつ x\notin B\}\]
である。ここで、$A\subset B$を仮定すると、$\forall x [ x\in A \Longrightarrow x\in B]$なので、$A-B=\emptyset$が成り立つ。逆に、$A-B=\emptyset$を仮定すると、$\forall x [ x\in A \Longrightarrow x\in B]$になるので、$A\subset B$が成り立つ。\\

\item 定義から
\begin{align*}
A\cup (B-A)&=\{x|x\in A または x\in (B-A)\}\\
&=\{x|x\in A または (x\in B かつ x\notin A)\}\\
&=\{x|x\in A または x\in B\}\\
&= A\cup B
\end{align*}
よって、1と本質的に同じ問題なので、成立する。\\
\item 定義から
\begin{align*}
B-(B-A)&=\{x| x\in B かつ x\notin (B-A)\}\\
&=\{x| x\in B かつ (x\notin B または x\in A)\}\\
&=\{x| x \in B かつ x\in A\}\\
&= A\cap B
\end{align*}
よって、本質的に2と同じ問題なので、成立する。
\end{enumerate}

\subsection{}
\begin{enumerate}
\item
\begin{align*}
(A\cup B) \cap (A\cup C) \cap(B\cup C) 
&=((A\cup B) \cap (A\cup C)) \cap(B\cup C) \\
&=((A\cup B) \cap A) \cup ((A\cup B) \cap C)) \cap(B\cup C) \\
&=(A\cup ((A\cap C) \cup (B\cap C)))\cap(B\cup C) \\
&=(A \cup (B\cap C))\cap(B\cup C) \\
&=(A\cap(B\cup C)) \cup ((B\cap C)\cap (B\cup C))\\
&=((A\cap B)\cup (A\cap C)) \cup (B\cap C)\\
&=(A\cap B)\cup (A\cap C) \cup (B\cap C)
\end{align*}
\item 1の結果を用いる。
\begin{align*}
&\quad(A\cup B) \cap (A\cup C) \cap(A\cup D)\cap (B\cup C) \cap (B\cup D) \cap(C\cup D)\\
&=((A\cup B) \cap (A\cup C)\cap(B\cup C) )\cap((A\cup D)\cap (B\cup D)\cap(C\cup D))\\
&=((A\cap B)\cup (A\cap C) \cup (B\cap C))\cap (D\cup (A\cap B \cap C))\\
&=(((A\cap B)\cup (A\cap C) \cup (B\cap C))\cap D)\cup(((A\cap B)\cup (A\cap C) \cup (B\cap C))\cap(A\cap B \cap C))\\
&=((A\cap B\cap D)\cup (A\cap C\cap D) \cup (B\cap C\cap D))\cup (A\cap B \cap C)\\
&=(A\cap B \cap C)\cup(A\cap B\cap D)\cup(A\cap C\cap D)\cup (B\cap C\cap D)
\end{align*}
\end{enumerate}


\section{ド・モルガンの法則}
\subsection{}
\begin{enumerate}
\item
\begin{align*}
(A^c)^c&=(X-A)^c\\
&=\{x | x\in X - (X-A)\}\\
&=\{x\in X|x \notin X-A\}\\
&=\{x\in X|「x \in X かつ x \notin A」ではない\}\\
&=\{x\in X| x\notin X または x \in A\}\\
&=\{x\in X | x \in A\}\\
&= A
\end{align*}


\item
\[X^c = X-X =\{x| x\in X かつ x\notin X\}= \emptyset \]

\item
\[\emptyset^c = X-\emptyset = \{x | x\in X かつ x\notin \emptyset\} = X\]

\item
\begin{align*}
A\cup A^c &= \{x| x\in A または x\in A^c\}\\
&=\{x \in X | x \in A または x\notin A\}\\
&=X
\end{align*}

\item
\[A\cap A^c = (A^c)^c \cap A^c = (A^c \cup A)^c = X^c = \emptyset\]

\item
\begin{align*}
A-B&=\{x\in X | x\in A かつ x\notin B\}\\
&=\{x | x\in A かつ x\in X-B\}\\
&=\{x| x\in A かつ x\in B^c\}\\
&= A\cap B^c
\end{align*}

\item

\[(A^c\cap B^c)^c = (A^c)^c \cup (B^c)^c = A\cup B\]

\end{enumerate}

\subsection{}
\begin{enumerate}
\item
\begin{align*}
(A\cup B)-C &= \{x | x \in A\cup B かつ x\notin C\}\\
&=\{x| (x \in A または x\in B) かつ x\notin C\}\\
&=\{x| (x\in A かつ x\notin C) または (x\in B かつ x\notin C)\}\\
&=\{x| x\in A-C または x\in B-C\}\\
&=(A-C)\cup (B-C)
\end{align*}

\item
\begin{align*}
(A\cap B)-C &= \{x | x \in A\cap B かつ x\notin C\}\\
&=\{x| (x \in A かつ x\in B) かつ x\notin C\}\\
&=\{x| (x\in A かつ x\notin C) かつ (x\in B かつ x\notin C)\}\\
&=\{x| x\in A-C かつ x\in B-C\}\\
&=(A-C)\cap (B-C)
\end{align*}

\end{enumerate}

\subsection{}
\begin{enumerate}
\item
\begin{enumerate}
\item
\begin{align*}
A\circ B &= (A-B)\cup(B-A)\\
&=(B-A)\cup (A-B)\\
&= B\circ A
\end{align*}


\item
\begin{align*}
(A\circ B)\circ C &= ((A\circ B)-C) \cup (C-(A\circ B))\\
&=(((A-B)\cup(B-A))-C)\cup(C-((A-B)\cup(B-A)))\\
&=(((A-B)-C)\cup((B-A)-C))\cup((C-(A-B)) \cap (C - (B-A)))\\
&=((A-(B\cup C)) \cup (B-(A\cup C))) \cup (((C-A)\cup (C\cap B))\cap((C-B)\cup(C\cap A)))\\
&=((A-(B\cup C)) \cup (B-(A\cup C))) \cup \\
&\qquad((C-A)\cap(C-B)) \cup( (C-A)\cap(C\cap A))\cup((C\cap B)\cap(C-B))\cup((C\cap B)\cap(C\cap A))\\
&=((A-(B\cup C)) \cup (B-(A\cup C)))\cup((C-(A\cup B)) \cup \emptyset \cup \emptyset \cup (A\cap B \cap C))\\
&=(A-(B\cup C)) \cup (B-(C\cup A))\cup(C-(A\cup B)) \cup  (A\cap B \cap C)
\end{align*}

ここで、
\begin{align*}
C-(A-B)&=\{x|x\in C かつ x\notin (A-B)\}\\
&=\{x|x\in C かつ (x\notin A または x\in B))\}\\
&=\{x|(x\in C かつ x\notin A) または (x\in C かつx\in B)\}\\
&=(C-A)\cup(C\cap B)
\end{align*}
という結果を、3行目から4行目への変形で用いた。また、
\begin{align*}
(A\cup B)\cap(C\cup D)&=(A\cap (C\cup D))\cup(B\cap (C\cup D))\\
&=(A\cap C)\cup (A\cap D) \cup (B\cap C)\cup(B\cap D)
\end{align*}
という結果を、4行目から(5,6行目)への変形で用いた。\\
さて、$(A\circ B)\circ C=(A-(B\cup C)) \cup (B-(C\cup A))\cup(C-(A\cup B)) \cup  (A\cap B \cap C)$と変形できるので、$(A\circ B)\circ C$は、A,B,Cが全く同等で、任意の二つを入れ替えても同じ値であることがわかる。よって、$(A\circ B)\circ C=A\circ (B\circ C)$。
\item
\[A\circ A= (A-A)\cup (A-A)=\emptyset\]

\item
\[A\circ \emptyset = (A-\emptyset)\cup(\emptyset-A)=A\]

\end{enumerate}
\item
\begin{align*}
&\qquad A\circ X=B\\
&\iff (A\circ X)-B=\emptyset かつB-(A\circ X)=\emptyset\\
&\iff ((A\circ X)-B) \cup (B-(A\circ X))=\emptyset\\
&\iff ((A\circ X)\circ B)=\emptyset\\
&\iff ((A\circ B)\circ X)=\emptyset\\
&\iff ((A\circ B)-X)\cup (X-(A\circ B))=\emptyset\\
&\iff ((A\circ B)-X)=\emptyset かつ (X-(A\circ B))=\emptyset\\
&\iff A\circ B=X
\end{align*}
ここで、$((A\circ X)\circ B)=\emptyset\iff((A\circ B)\circ X)=\emptyset$という変形には、1の結果を用いた。
以上より、集合A,Bを任意に与えたとき、$A\circ X=B$を満足する集合$X$が$A\circ B$と表せるので、$X$はただ一つ存在することが示せた。
\end{enumerate}

\section{直積集合}
\subsection{}
\begin{enumerate}
\item
\begin{align*}
A\times(B\cup C)&= \{(x,y)|x\in A かつy\in(B\cup C)\}\\
&=\{(x,y)|(x\in A かつ y\in B)または(x\in A かつ y\in C)\}\\
&=\{(x,y)|(x,y)\in (A\times B) または (x,y)\in(A\times C)\}\\
&=(A\times B) \cup (A\times C)
\end{align*}

\item
\begin{align*}
A\times(B\cap C)&= \{(x,y)|x\in A かつy\in(B\cap C)\}\\
&=\{(x,y)|(x\in A かつ y\in B)かつ(x\in A かつ y\in C)\}\\
&=\{(x,y)|(x,y)\in (A\times B) かつ (x,y)\in(A\times C)\}\\
&=(A\times B) \cap (A\times C)
\end{align*}

\item
\begin{align*}
(A\cup B)\times C&=\{(x,y)|x\in (A\cup B) かつ y\in C\}\\
&=\{(x,y)|(x\in A かつ y\in C)または(x\in B かつ y\in C)\}\\
&=\{(x,y)|(x,y)\in A\times C または (x,y)\in B \times C\}\\
&=(A\times C)\cup(B\times C)
\end{align*}

\item
\begin{align*}
(A\cap B)\times C&=\{(x,y)|x\in (A\cap B) かつ y\in C\}\\
&=\{(x,y)|(x\in A かつ y\in C)かつ(x\in B かつ y\in C)\}\\
&=\{(x,y)|(x,y)\in A\times C かつ (x,y)\in B \times C\}\\
&=(A\times C)\cap(B\times C)
\end{align*}

\end{enumerate}

\subsection{}
\begin{align*}
(X\times Y)-(A\times B)&=\{(s,t)|(s,t)\in(X\times Y)かつ(s,t)\notin(A\times B)\}\\
&=\{(s,t)|(s\in(X-A) かつ t\in Y)\\
&\qquad\qquad または(s\in X かつ t\in(Y-B))\}\\
&=\{(s,t)|(s,t)\in ((X-A)\times Y) または (s,t)\in(X\times (Y-B))\}\\
&=((X-A)\times Y)\cup(X\times(Y-B))
\end{align*}

\section{写像}
\subsection{}
\begin{enumerate}
\item
\[(f\circ g)(x)=f(x^2+1)=x^2+3\]
\item
\[(g\circ f)(x)=g(x+2)=(x+2)^2+1=x^2+4x+5\]
\item
\[(f\circ f)(x)=f(x+2)=x+4\]
\item
\[(g\circ g)(x)=g(x^2+1)=(x^2+1)^2+1=x^4+2x^2+2\]
\end{enumerate}

\subsection{}
\begin{enumerate}
\item
\begin{align*}
(\bigcup_{\lambda\in\Lambda}A_\lambda)\cap B&=\{x| x\in\bigcup_{\lambda\in\Lambda}A_\lambda かつ x\in B\}\\
&=\{x|\exists \lambda\in\Lambda[x\in A_{\lambda}]かつx\in B\}\\
&=\{x|\exists \lambda\in\Lambda[x\in A_{\lambda} かつ x\in B]\}\\
&=\{x|\exists \lambda\in\Lambda[x\in (A_{\lambda} \cap B)]\}\\
&=\bigcup_{\lambda\in\Lambda}(A_{\lambda}\cap B)
\end{align*}

\item
\begin{align*}
(\bigcap_{\lambda\in\Lambda}A_\lambda)\cup B&=\{x| x\in\bigcap_{\lambda\in\Lambda}A_\lambda または x\in B\}\\
&=\{x|\forall \lambda\in\Lambda[x\in A_{\lambda}]またはx\in B\}\\
&=\{x|\forall \lambda\in\Lambda[x\in A_{\lambda} または x\in B]\}\\
&=\{x|\forall \lambda\in\Lambda[x\in (A_{\lambda} \cup B)]\}\\
&=\bigcap_{\lambda\in\Lambda}(A_{\lambda}\cup B)
\end{align*}

\end{enumerate}

\end{document}

