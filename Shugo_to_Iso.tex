\documentclass{jsarticle}


\usepackage{ascmac}
\usepackage[top=30truemm,bottom=30truemm,left=25truemm,right=25truemm]{geometry}
\usepackage{amsfonts}
\usepackage{amsmath,amssymb}
\begin{document}

\title{集合と位相}
\author{nukui}
\date{\today}
\maketitle

\part{集合と写像}
\section{集合とは}
\subsection{}
\begin{enumerate}
\item 成り立つ。$\because$ Yに含まれる要素は全てXに含まれる。
\item 成り立つ。$\because$3はWに含まれるがZに含まれない。
\item 成り立つ。$\because$4はVに含まれるが、Yに含まれない。
\item 成り立たない。$\because$4はVに含まれるがXには含まれない。
\item 成り立たない。$\because$ 1はXに含まれるがWに含まれない。
\item 成り立たない。$\because$ Vの全ての要素はWに含まれる。
\item 成り立つ。$\because$ Vの全ての要素はZに含まれる。
\item 成り立つ。 $\because$ 3はXに含まれるがZに含まれない。
\item 成り立たない。 $\because$ Yに含まれる全ての要素はZに含まれる。
\item 成り立たない。$\because$ 3はWに含まれるがYには含まれない。
\end{enumerate}

\subsection{}
\begin{enumerate}
\item D
\item B
\item C,E,F
\item B,D
\end{enumerate}

\subsection{}
\begin{enumerate}
\item 成り立たない。
\item 成り立つ。
\item 成り立つ。
\item 成り立つ。
\item 成り立たない。
\item 成り立つ。
\end{enumerate}

\subsection{}
集合Aが1個の元から成るとき、部分集合はAと$\emptyset$の2通り。よって$n=1$のとき、命題は成り立つ。\\
集合Aがn個の元から成り、その部分集合は全部で$2^n$個から成るとする。
今、集合Aに元Xを一つ加え、n+1個の元から成る集合Bを考える。
集合Bの部分集合は、
\begin{enumerate}
\item 集合Aの部分集合と一致。($2^n$個)
\item 集合Aの部分集合に元Xを加えたものに一致。($2^n$個)
\end{enumerate}
のいずれかである。よって、集合Bの部分集合の個数は$2^n+2^n=2^{n+1}$個になる。以上より、すべての自然数$n$で命題は成り立つ。

\section{集合の演算}
\subsection{}
意味を考えれば、確かに成り立つことがわかる。

\subsection{}
\begin{enumerate}
\item
\begin{align*}
(A-B)\cup(A\cap B)&=\{x|x\in(A-B) またはx\in(A\cap B)\}\\
&=\{x|(x\in A かつ x\notin B) または (x\in A かつ x\in B)\}\\
&=\{x|x\in A かつ (x\notin B または x\in B)\}\\
&=\{x|x\in A\}\\
&=A
\end{align*}


\item
\begin{align*}
(A-B)\cup B &=\{x| x\in (A-B) または x \in B\}\\
&=\{ x | (x \in A かつ x\notin B) または x\in B \}\\
&=\{x | (x\in Aまたは x\in B) かつ( x\notin B または x\in B)\}\\
&=\{x| (x\in Aまたは x\in B)\}\\
&=A\cup B
\end{align*}


\item
\begin{align*}
B\cap(A-B)&=\{x|x\in B かつ x \in(A-B)\}\\
&=\{x|x\in B かつ( x \in A かつ x\notin B)\}\\
&=\emptyset
\end{align*}
\end{enumerate}

\subsection{}
\begin{enumerate}
\item
$A_1 \subset A$を仮定する。
\[x \in A_1 かつ x\notin B \Longrightarrow x\in A かつ x\notin B\]
なので、$x\in A_1-B$とすると、$x\in A-B$が示せる。つまり、$A_1-B \subset A-B$。
\item
$B_1 \subset B$を仮定する。
\[x \in A かつ x\notin B \Longrightarrow x\in A かつ x\notin B_1\]
なので、$x\in A-B$とすると、$x\in A-B_1$が示せる。つまり、$A-B \subset A-B_1$。
\end{enumerate}

\subsection{}
\begin{align*}
A-B&=\{x|x\in A かつ x\notin B\}\\
&=\{x|x\in A かつ (x\notin A または x\notin B)\}\\
&=\{x|x\in A かつ x\notin A\cap B\}
\end{align*}
\[A=\{x|x\in A\}\]
なので、
\[A-B=A \Longleftrightarrow  A-B \supset A \Longleftrightarrow A\cap B = \emptyset\]

\end{document}

