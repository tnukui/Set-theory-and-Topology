\documentclass{jsarticle}


\usepackage{ascmac}
\usepackage[top=30truemm,bottom=30truemm,left=25truemm,right=25truemm]{geometry}
\usepackage{amsfonts}
\usepackage{amsmath,amssymb}
\begin{document}

\title{集合と位相}
\author{nukui}
\date{\today}
\maketitle

\part{集合と写像}
\section{集合とは}
\subsection{}
\begin{enumerate}
\item 成り立つ。$\because$ Yに含まれる要素は全てXに含まれる。
\item 成り立つ。$\because$3はWに含まれるがZに含まれない。
\item 成り立つ。$\because$4はVに含まれるが、Yに含まれない。
\item 成り立たない。$\because$4はVに含まれるがXには含まれない。
\item 成り立たない。$\because$ 1はXに含まれるがWに含まれない。
\item 成り立たない。$\because$ Vの全ての要素はWに含まれる。
\item 成り立つ。$\because$ Vの全ての要素はZに含まれる。
\item 成り立つ。 $\because$ 3はXに含まれるがZに含まれない。
\item 成り立たない。 $\because$ Yに含まれる全ての要素はZに含まれる。
\item 成り立たない。$\because$ 3はWに含まれるがYには含まれない。
\end{enumerate}

\subsection{}
\begin{enumerate}
\item D
\item B
\item C,E,F
\item B,D
\end{enumerate}

\subsection{}
\begin{enumerate}
\item 成り立たない。
\item 成り立つ。
\item 成り立つ。
\item 成り立つ。
\item 成り立たない。
\item 成り立つ。
\end{enumerate}

\subsection{}
集合Aが1個の元から成るとき、部分集合はAと$\emptyset$の2通り。よって$n=1$のとき、命題は成り立つ。\\
集合Aがn個の元から成り、その部分集合は全部で$2^n$個から成るとする。
今、集合Aに元Xを一つ加え、n+1個の元から成る集合Bを考える。
集合Bの部分集合は、
\begin{enumerate}
\item 集合Aの部分集合と一致。($2^n$個)
\item 集合Aの部分集合に元Xを加えたものに一致。($2^n$個)
\end{enumerate}
のいずれかである。よって、集合Bの部分集合の個数は$2^n+2^n=2^{n+1}$個になる。以上より、すべての自然数$n$で命題は成り立つ。





\end{document}

